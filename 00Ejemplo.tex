\documentclass[11pt]{article}

%%%%%%%%%%%%%%%%%%%%%%%%%%%%%%%%%%%%%%%%
%% Paquetes
%%%%%%%%%%%%%%%%%%%%%%%%%%%%%%%%%%%%%%%%
%% Paquetes
\usepackage[spanish,es-nolayout]{babel}
\usepackage[T1]{fontenc}
\usepackage[utf8]{inputenc}
\usepackage{arev}
\usepackage{xcolor}
\usepackage{amsmath}
\usepackage{booktabs}
\usepackage{graphicx}
\usepackage[a4paper,left=3.5cm,right=2.0cm,top=1.2cm,
            headsep=3mm,headheight=2.3cm,includehead,
            footskip=15mm]{geometry}
\usepackage{float}
\usepackage{listings}
    \renewcommand{\lstlistingname}{Código}
\usepackage{setspace}
    \setstretch{1.15}
\usepackage{enumitem}
\usepackage[colorlinks,allcolors=.,breaklinks]{hyperref}
\usepackage{silence}

\WarningFilter{latexfont}{Font shape `TS1/fav/m/n' undefined}
\WarningFilter{latexfont}{Some font shapes were not available, defaults substituted.}

%%%%%%%%%%%%%%%%%%%%%%%%%%%%%%%%%%%%%%%%
%% Colores
\color[HTML]{000078}
\definecolor{celeste}{HTML}{73EDFF}
\definecolor{gris}{HTML}{F0F0F0}


\makeatletter

%%%%%%%%%%%%%%%%%%%%%%%%%%%%%%%%%%%%%%%%
%% Encabezados y pies de página
\usepackage{fancyhdr}
    \fancyhf{}
    \renewcommand{\headrulewidth}{0pt}
    \fancyhead[L]{\hspace*{-2.5cm}
    \includegraphics[width=18cm]{Figuras/encabezado.png}
    }
    \fancyfoot[R]{\footnotesize
    \begin{tabular}{@{}l@{\ \ }l@{\ \ }l@{}}
    \textcolor[HTML]{73EDFF}{\rule{6.7cm}{0.9mm}} &
    \textcolor[HTML]{73EDFF}{\rule{5.7cm}{0.9mm}} &
    \textcolor[HTML]{73EDFF}{\rule{2.7cm}{0.9mm}} \\
       \@autorcorto  & \@fecha & \textbf{pág. \thepage}
    \end{tabular}
    }
\pagestyle{fancy}

%%%%%%%%%%%%%%%%%%%%%%%%%%%%%%%%%%%%%%%%
%% Datos
\newcommand{\@asignatura}{}
\newcommand{\@titulo}{}
\newcommand{\@subtitulo}{}
\newcommand{\@autor}{}
\newcommand{\@autorcorto}{}
\newcommand{\@fecha}{\today}

\newcommand{\asignatura}[1]{
    \renewcommand{\@asignatura}{#1}}
\newcommand{\titulo}[1]{
    \renewcommand{\@titulo}{#1}}
\newcommand{\subtitulo}[1]{
    \renewcommand{\@subtitulo}{#1}}
\newcommand{\autor}[2][]{\ifthenelse{\equal{#1}{}}
   {\renewcommand\@autorcorto{#2}\renewcommand\@autor{#2}}
   {\renewcommand\@autorcorto{#1}\renewcommand\@autor{#2}}}
\newcommand{\fecha}[1]{
    \renewcommand{\@fecha}{#1}}

%%%%%%%%%%%%%%%%%%%%%%%%%%%%%%%%%%%%%%%%
%% Portada
\newcommand{\portada}{
    \newgeometry{margin=12mm}
    \thispagestyle{empty}
    \noindent
    \fcolorbox{celeste}{celeste}{
    \begin{minipage}[t][10cm][t]{18.1cm}
    \flushleft
    \hspace{0pt}\\
    {\fontsize{45pt}{45pt}\selectfont \textbf{\@asignatura} \\}
    \vspace{\fill}
    {\fontsize{35pt}{35pt}\selectfont \@titulo \\[1cm]\hspace{0pt}}
    \end{minipage}
    }\\[9mm]
    \includegraphics[width=6cm]{Figuras/portada.png}
    \hspace{7mm}
    \fcolorbox{gris}{gris}{
    \begin{minipage}[b][15.59cm][t]{11.2cm}
    \flushleft
    \hspace{0pt}\\
    \Huge
    \textbf{\@autor}\\[1.5cm]
    \LARGE
    \@fecha
    \end{minipage}
    }
    
    \clearpage
    \restoregeometry
    
    \noindent
    \parbox{\linewidth}{\Huge\flushleft
    {\bfseries \@titulo}
    \ifthenelse{\equal{\@subtitulo}{}}{}{\\ \@subtitulo}\\[1cm]
    }
}

%%%%%%%%%%%%%%%%%%%%%%%%%%%%%%%%%%%%%%%%
%% Bibliografía
\usepackage[babel]{csquotes}
\usepackage[style=apa,backend=biber]{biblatex}
    \DeclareLanguageMapping{spanish}{spanish-apa}
    
    \DefineBibliographyStrings{spanish}{%
      andothers = {et al\adddot},
    }
    
    \DefineBibliographyExtras{spanish}
        {\setcounter{smartand}{1}% or some other value
         \let\lbx@finalnamedelim=\lbx@es@smartand
         \let\lbx@finallistdelim=\lbx@es@smartand}
    
    \setlength{\bibhang}{\parindent}  

%%%%%%%%%%%%%%%%%%%%%%%%%%%%%%%%%%%%%%%%
%% Leyendas
\usepackage[font={small},labelfont={bf,small},
  justification=centerlast,tablename=Tabla]{caption}

%%%%%%%%%%%%%%%%%%%%%%%%%%%%%%%%%%%%%%%%
%% Código

\definecolor{codegreen}{rgb}{0,0.6,0}
\definecolor{codegray}{rgb}{0.5,0.5,0.5}
\definecolor{codepurple}{rgb}{0.58,0,0.82}

\lstdefinestyle{mystyle}{
    backgroundcolor=\color{celeste!20},   
    commentstyle=\color{codegreen},
    keywordstyle=\color{magenta},
    numberstyle=\tiny\color{codegray},
    stringstyle=\color{codepurple},
    basicstyle=\ttfamily\footnotesize,
    breakatwhitespace=false,         
    breaklines=true,                 
    captionpos=b,                    
    keepspaces=true,                 
    numbers=left,                    
    numbersep=5pt,                  
    showspaces=false,                
    showstringspaces=false,
    showtabs=false,                  
    tabsize=2,
    linewidth=0.98\linewidth,
    xleftmargin=0.5cm
}

\lstset{style=mystyle}


\makeatother

\usepackage{lipsum} % borrar este paquete


%%%%%%%%%%%%%%%%%%%%%%%%%%%%%%%%%%%%%%%%
%% Información
\asignatura{Fundamentos de la ciencia de datos}
\titulo{PEC1: ¿Ciencia en los datos?}
\subtitulo{La Ciencia de Datos en Entornos Deportivos}
\autor{Andrés Esteban Merino Toapanta}
\fecha{\today}

%%%%%%%%%%%%%%%%%%%%%%%%%%%%%%%%%%%%%%%%
%% Bibliografía
\addbibresource{00ejemplo.bib}

%%%%%%%%%%%%%%%%%%%%%%%%%%%%%%%%%%%%%%%%
%% Nuevos comandos


%%%%%%%%%%%%%%%%%%%%%%%%%%%%%%%%%%%%%%%%
\begin{document}


\portada

%%%%%%%%%%%%%%%%%%%%%%%%%%%%%%%%%%%%%%%%
\section{Introducción}
%%%%%%%%%%%%%%%%%%%%%%%%%%%%%%%%%%%%%%%%

\lipsum[1-5]

%%%%%%%%%%%%%%%%%%%%%%%%%%%%%%%%%%%%%%%%
\section{Imágenes}
%%%%%%%%%%%%%%%%%%%%%%%%%%%%%%%%%%%%%%%%

Un tamaño recomendado es un ancho del 75\% de la línea de texto con una altura proporcional a la primera. Todas las imágenes deben incluir una leyenda.

\begin{figure}[H]
    \centering
    \includegraphics[width=0.60\textwidth]{Figuras/encabezado.png}
    \caption{Leyenda de la figura.}
    \label{fig:etiqueta de la figura}
\end{figure}


%%%%%%%%%%%%%%%%%%%%%%%%%%%%%%%%%%%%%%%%
\section{Tablas}
%%%%%%%%%%%%%%%%%%%%%%%%%%%%%%%%%%%%%%%%

Para la composición de tablas, la letra siempre debe ser de tamaño menor a la del resto del texto y se recomienda optar por el siguiente formato:

\begin{table}[H]
    \centering\small
    \begin{tabular}{ccc}
    \toprule
        \textbf{Fórmula} & \textbf{Prueba 1} & \textbf{Prueba 2} \\ 
    \midrule
        Compuesto 1 & $38.4$  &  $6.32$\\ 
        Compuesto 2 & $16.6$ & $12.5$ \\ 
    \bottomrule
    \end{tabular}
    \label{tab:01}
    \caption{Resultados de la experimentación de distintas substancias.}
\end{table}


%%%%%%%%%%%%%%%%%%%%%%%%%%%%%%%%%%%%%%%%
\section{Código}
%%%%%%%%%%%%%%%%%%%%%%%%%%%%%%%%%%%%%%%%

Ejemplo de código.

\begin{lstlisting}[language=Python,caption={Ejemplo de código.},captionpos=b]
def OrdenBurbuja(a):
    for i in range(len(a)-2):
        for j in range(len(a)-i-1):
            if a[j] > a[j+1]:
                a[j],a[j+1] = a[j+1],a[j]
    return a
\end{lstlisting}

%%%%%%%%%%%%%%%%%%%%%%%%%%%%%%%%%%%%%%%%
\section{Bibliografía y citas}
%%%%%%%%%%%%%%%%%%%%%%%%%%%%%%%%%%%%%%%%

La bibliografía debe incluirse mediante un archivo \texttt{.bib} con el mismo nombre que el archivo principal. El estilo bibliográfico a usar es APA séptima edición. Para las citas puede utilizar los siguientes comandos según sea adecuado:
\begin{itemize}
\item 
    Cita completa entre paréntesis \verb"\parencite{ }": \parencite{Bib06}
\item
    Cita completa sin paréntesis \verb"\textcite{ }": \textcite{Bib06}
\item 
    Cita completa entre paréntesis \verb"\cite{ }": \cite{Bib06}
\item
    Cita de autor \verb"\citeauthor{ }": \citeauthor{Bib06}
\item
    Cita de año \verb"\citeyear{ }": \citeyear{Bib06}
\item 
    Cita con opciones extras \verb"\parencite[ ][ ]{ }": \parencite[ver][pág. 66]{Bib06}
\end{itemize}


%%%%%%%%%%%%%%%%%%%%%%%%%%%%%%%%%%%%%%%%
%% Referencias
%%%%%%%%%%%%%%%%%%%%%%%%%%%%%%%%%%%%%%%%
\nocite{*}
\printbibliography



\end{document}
